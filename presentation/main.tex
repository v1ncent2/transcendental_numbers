%%%%%%%%%%%%%%%%%%%%%%%%%%%%%%%%%%%%%%%%%
% Beamer Presentation
% LaTeX Template
% Version 2.0 (March 8, 2022)
%
% This template originates from:
% https://www.LaTeXTemplates.com
%
% Author:
% Vel (vel@latextemplates.com)
%
% License:
% CC BY-NC-SA 4.0 (https://creativecommons.org/licenses/by-nc-sa/4.0/)
%
%%%%%%%%%%%%%%%%%%%%%%%%%%%%%%%%%%%%%%%%%

%----------------------------------------------------------------------------------------
%	PACKAGES AND OTHER DOCUMENT CONFIGURATIONS
%----------------------------------------------------------------------------------------

\documentclass{beamer}

\usetheme{Madrid}


\title{Transcendental Numbers} 

\author{Vincent Lin} 

\date{\today} 

%----------------------------------------------------------------------------------------

\begin{document}

%----------------------------------------------------------------------------------------
%	TITLE SLIDE
%----------------------------------------------------------------------------------------

\begin{frame}
    \titlepage{}
\end{frame}


%----------------------------------------------------------------------------------------
%	PRESENTATION BODY SLIDES
%----------------------------------------------------------------------------------------

\begin{frame}
	\frametitle{Introduction}
	Motivation, definitions, etc

    \begin{definition}
        An \alert{algebraic number} is $\alpha \in \mathbb{C}$ which is a root of a nonzero polynomial in $\mathbb{Q}[x]$. $\alpha \in \mathbb{C}$ that are not algebraic numbers are called \alert{transcendental numbers}.
    \end{definition}

 	\begin{example}
		\begin{itemize}
		    \item{$q \in \mathbb{Q}$ are algebraic numbers.}
		\end{itemize}
	\end{example} 
\end{frame}

\begin{frame}
    \frametitle{More Definitions}
\end{frame}

\begin{frame}
    \frametitle{Can we construct a transcendental number?}
    \begin{theorem}
        $\sum\limits_{i = 1}^{\infty}\frac{1}{10^{n!}}$ is transcendental.
    \end{theorem}
\end{frame}
\begin{frame}
	\frametitle{Which of the following are transcendental?}
	
    \[\log(2), \log_{2}(21441)\] \\ \[e, \pi, \sqrt[4]{\pi}\] \\ \[\cos{1}\] \[2^{\sqrt{2}}, {\sqrt{2}}^{\sqrt{2}}, {i}^{i}\] \\ \[e + \pi, e\pi, e^{\pi}, {\pi}^{e}, {e}^{e}\] \\ \[\phi, \gamma, \lambda\]
	
\end{frame}


\begin{frame}
    \frametitle{Transcendental by the Lindemann-Weierstrass Theorem}
    \[\alert{\log(2)}, \log_{2}(21441)\] \\ \[\alert{e}, \alert{\pi}, \alert{\sqrt[4]{\pi}}\] \\ \[\alert{\cos{1}}\] \\ \[2^{\sqrt{2}}, {\sqrt{2}}^{\sqrt{2}}, {i}^{i}\] \\ \[e + \pi, e\pi, e^{\pi}, {\pi}^{e}, {e}^{e}\] \\ \[\phi, \gamma, \lambda\]
\end{frame}

\begin{frame}
	\frametitle{Transcendental by the Gelfond-Schneider Theorem}
	
    \[\log(2), \alert{\log_{2}(21441)}\] \\ \[e, \pi, \sqrt[4]{\pi}\] \\ \[\cos{1}\] \[\alert{2^{\sqrt{2}}}, \alert{{\sqrt{2}}^{\sqrt{2}}}, \alert{{i}^{i}}\] \\ \[e + \pi, e\pi, \alert{e^{\pi}}, {\pi}^{e}, {e}^{e}\] \\ \[\phi, \gamma, \lambda\]
	
\end{frame}

\begin{frame}
	\frametitle{Unknown. Schanuel's Conjecture implies transcendence.}

    \[\log(2), \log_{2}(21441)\] \\ \[e, \pi, \sqrt[4]{\pi}\] \\ \[\cos{1}\] \[2^{\sqrt{2}}, {\sqrt{2}}^{\sqrt{2}}, {i}^{i}\] \\ \[\alert{e + \pi}, \alert{e\pi}, e^{\pi}, \alert{{\pi}^{e}}, \alert{{e}^{e}}\] \\ \[\phi, \gamma, \lambda\]
   
\end{frame}

\begin{frame}
	\frametitle{Algebraic. Minimal polynomial is $x^2 - x - 1$.}
	
    \[\log(2), \log_{2}(21441)\] \\ \[e, \pi, \sqrt[4]{\pi}\] \\ \[\cos{1}\] \[2^{\sqrt{2}}, {\sqrt{2}}^{\sqrt{2}}, {i}^{i}\] \\ \[e + \pi, e\pi, e^{\pi}, {\pi}^{e}, {e}^{e}\] \\ \[\alert{\phi}, \gamma, \lambda\]
	
\end{frame}

\begin{frame}
	\frametitle{Both irrationality and transcendence are unknown.}
	
    \[\log(2), \log_{2}(21441)\] \\ \[e, \pi, \sqrt[4]{\pi}\] \\ \[\cos{1}\] \[2^{\sqrt{2}}, {\sqrt{2}}^{\sqrt{2}}, {i}^{i}\] \\ \[e + \pi, e\pi, e^{\pi}, {\pi}^{e}, {e}^{e}\] \\ \[\phi, \alert{\gamma}, \lambda\]
	
\end{frame}

\begin{frame}
	\frametitle{Algebraic. Minimal polynomial has degree 71!}
	
    \[\log(2), \log_{2}(21441)\] \\ \[e, \pi, \sqrt[4]{\pi}\] \\ \[\cos{1}\] \[2^{\sqrt{2}}, {\sqrt{2}}^{\sqrt{2}}, {i}^{i}\] \\ \[e + \pi, e\pi, e^{\pi}, {\pi}^{e}, {e}^{e}\] \\ \[\phi, \gamma, \alert{\lambda}\]
	
\end{frame}

\begin{frame}
    \frametitle{Lindemann-Weierstrass Theorem}
    \begin{theorem}
        If $\alpha_1, \ldots, \alpha_n$ are algebraic numbers and linearly independent over $\mathbb{Q}$, then $e^{\alpha_1}, \ldots, e^{\alpha_n}$ are algebraically independent over $\mathbb{Q}$.
    \end{theorem}
    \begin{corollary}
        If $\alpha$ is a nonzero algebraic number, then $e^{\alpha}$ is transcendental.
    \end{corollary}
\end{frame}

\begin{frame}
    \frametitle{Gelfond-Schneider Theorem}
    \begin{theorem}
        Let $\alpha$ be a nonzero algebraic not equal to $1$. If $\beta$ is not rational ($\in \mathbb{C} \setminus \mathbb{Q}$), then $\alpha^{\beta}$ is transcendental.
    \end{theorem}
\end{frame}

\begin{frame}
    \frametitle{Baker's Theorem}
    \begin{theorem}
        
    \end{theorem}
\end{frame}

\begin{frame}
    \frametitle{Schanuel's Conjecture}
    \begin{theorem}
        
    \end{theorem}
\end{frame}



%------------------------------------------------

\begin{frame} % Use [allowframebreaks] to allow automatic splitting across slides if the content is too long
	\frametitle{References}
	
	\begin{thebibliography}{99} % Beamer does not support BibTeX so references must be inserted manually as below, you may need to use multiple columns and/or reduce the font size further if you have many references
		\footnotesize % Reduce the font size in the bibliography
		
		\bibitem[Smith, 2022]{p1}
			John Smith (2022)
			\newblock Publication title
			\newblock \emph{Journal Name} 12(3), 45 -- 678.
			
		\bibitem[Kennedy, 2023]{p2}
			Annabelle Kennedy (2023)
			\newblock Publication title
			\newblock \emph{Journal Name} 12(3), 45 -- 678.
	\end{thebibliography}
\end{frame}




%----------------------------------------------------------------------------------------

\end{document} 
