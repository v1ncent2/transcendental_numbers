%%%%%%%%%%%%%%% Page Setup %%%%%%%%%%%%%%%

\documentclass[a4paper, 11pt]{book}
\usepackage[utf8]{inputenc}
\usepackage[framemethod=tikz]{mdframed}
\usepackage[margin=3cm]{geometry}
\setlength\parindent{0pt}
\setlength{\parskip}{\baselineskip}%
\usepackage{graphicx}
\usepackage{tgadventor}
%\renewcommand*\familydefault{\sfdefault}
\usepackage[T1]{fontenc}
\usepackage{bm}

%%%%%%%%%%%%%%% Math Stuff %%%%%%%%%%%%%%%

\usepackage{amsmath}
\usepackage{amssymb}
\usepackage{tikz}
\usepackage{tikz-cd}
\usepackage{tkz-euclide}
\usepackage{pgf-pie}
\usepackage{cancel}

%%%%%%%%%%%%%%% Quotation Stuff %%%%%%%%%%%%%%%

\usepackage[english]{babel}
\usepackage[autostyle]{csquotes}

%%%%%%%%%%%%%%% Color %%%%%%%%%%%%%%%

\definecolor{p}{rgb}{0.9118, 0.4961, 0.6804}
\definecolor{p1}{rgb}{1, 0.92, 0.963}
\definecolor{g}{rgb}{0.023, 0.477, 0.219}
\definecolor{g1}{rgb}{0.03, 0.75, 0.20}
\definecolor{g2}{rgb}{0.95, 1, 0.95}

%%%%%%%%%%%%%%% Link Setup %%%%%%%%%%%%%%%

\usepackage{hyperref}
\hypersetup{colorlinks=true, linkcolor=g, filecolor=g, urlcolor=g,}

%%%%%%%%%%%%%%% Box %%%%%%%%%%%%%%%

\newmdenv[innerlinewidth=2pt, roundcorner=4pt,linecolor=g1,innerleftmargin=20pt,
innerrightmargin=20pt,innertopmargin=20pt,innerbottommargin=20pt,backgroundcolor = g2]{mybox}

%%%%%%%%%%%%%%% Border Stuff %%%%%%%%%%%%%%%

\usepackage{fancyhdr}
\pagestyle{fancy}
\fancyhf{}
\fancyhead[LE,RO]{\leftmark}
\fancyhead[RE,LO]{\thepage}
\fancyfoot[CE,CO]{}
\fancyfoot[LE,RO]{}

%%%%%%%%%%%%%%% Bug Fix Adjustments %%%%%%%%%%%%%%%

\setlength{\headheight}{14pt}
\setlength{\footskip}{55pt}

%%%%%%%%%%%%%%% Index %%%%%%%%%%%%%%%

\usepackage{makeidx}
\makeindex

%%%%%%%%%%%%%%% Custom Commands %%%%%%%%%%%%%%%

\def\greenlozenge{\mathbin{\color{g}\blacklozenge}}

\newcommand{\emphasis}[1]{\underline{\textbf{#1}} }
\newcommand{\vocab}[1]{\underline{\textbf{#1}}\index{#1}}

\newcommand{\claim}{\textbf{\underline{Claim}} }
%\newcommand{\corollary}{\underline{\textbf{Corollary}} }
\newcommand{\defn}{\underline{\textbf{Def}} }
\newcommand{\digression}{\underline{\textbf{Digression}} }
\newcommand{\easy}{\underline{\textbf{Easy}} }
\newcommand{\example}{\underline{\textbf{Example}} }
\newcommand{\fact}{\underline{\textbf{Fact}} }
\newcommand{\facts}{\underline{\textbf{Facts}} }
\newcommand{\goal}{\underline{\textbf{Goal}} }
\newcommand{\idea}{\underline{\textbf{Idea}} }
%\newcommand{\lemma}{\underline{\textbf{Lemma}} }
\newcommand{\need}{\underline{\textbf{Need}} }
\newcommand{\note}{\underline{\textbf{Note}} }
\newcommand{\proof}{\underline{\textbf{Proof}} }
\newcommand{\recall}{\underline{\textbf{Recall}} }
\newcommand{\remark}{\underline{\textbf{Remark}} }
%\newcommand{\theorem}{\underline{\textbf{Theorem}} }
\newcommand{\verify}{\underline{\textbf{Verify}} }


%%%%%%%%%%%%%%% New Theorems %%%%%%%%%%%%%%%


\newtheorem{theorem}{Theorem}[section]
\newtheorem{corollary}{Corollary}[theorem]
\newtheorem{lemma}[theorem]{Lemma}
\newtheorem{conjecture}[theorem]{Conjecture}

%%%%%%%%%%%%%%% Math Operators %%%%%%%%%%%%%%%
\DeclareMathOperator{\A}{\mathbb{A}}
\DeclareMathOperator{\Aut}{Aut}
\DeclareMathOperator{\C}{\mathbb{C}}
\DeclareMathOperator{\characteristic}{char}
\DeclareMathOperator{\cod}{cod}
\DeclareMathOperator{\dom}{dom}
\DeclareMathOperator{\Fix}{Fix}
\DeclareMathOperator{\Frac}{Frac}
\DeclareMathOperator{\Free}{Free}
\DeclareMathOperator{\id}{id}
\DeclareMathOperator{\Ima}{Im}
\DeclareMathOperator{\Mor}{Mor}
\DeclareMathOperator{\N}{\mathbb{N}}
\DeclareMathOperator{\Obj}{Obj}
\DeclareMathOperator{\Q}{\mathbb{Q}}
\DeclareMathOperator{\R}{\mathbb{R}}
\DeclareMathOperator{\Res}{Res}
\DeclareMathOperator{\Stab}{Stab}
\DeclareMathOperator{\Span}{span}
\DeclareMathOperator{\Spec}{Spec}
\DeclareMathOperator{\trdeg}{trdeg}
\DeclareMathOperator{\Z}{\mathbb{Z}}
%%%%%%%%%%%%%%% Title %%%%%%%%%%%%%%%

\title{Transcendental Numbers}
\author{Vincent Lin}

\begin{document}
\defn{Let $z \in \C$. The \vocab{Lambert $W$ function} is a multivalued function denoted $W(z)$ and is equal to all $w \in \C$ where $we^{w} = z$.\par}

Let the function $\Gamma: \{z : \mathfrak{R}(z) > 0\} \to \C$ via \[\Gamma(z) = \int_{0}^{\infty} t^{z-1}e^{-t} dt.\] 

\defn{The meromorphic \vocab{gamma function} is the analytic continuation of $\Gamma$ defined above and its poles are zero and the negative integers.}

\begin{mybox}
    \theorem{For $z \in \C$ with $\mathfrak{R}(z) > 0$, $\Gamma(z+1) = z\Gamma(z)$.}
\end{mybox}

\proof{Using integration by parts, we get that 
\begin{align*}
    \Gamma(z+1) &= \int_{0}^{\infty} t^{z}e^{-t} dt \\
                &= {\left[-t^{z}{e}^{-t}\right]}_{0}^{\infty} + \int_{0}^{\infty} z{t}^{z-1}e^{-t} dt \\
                &= \lim\limits_{t \to \infty} (-t^{z}e^{-t}) + z\int_{0}^{\infty} t^{z-1}e^{-t} dt \\
                &= 0 + z\int_{0}^{\infty} t^{z-1}e^{-t} dt \\
                &= z\Gamma(z)
\end{align*} \[\greenlozenge\]}

\note{Note that at $z = 1$, the function evaluates to \[\int_{0}^{\infty} e^{-t} dt = 1.\] Using this and the recurrence relation from before, we know \[\Gamma(n) = (n-1)!\] for positive integers $n$. Also, note that $\Gamma(\frac{1}{2}) = \sqrt{\pi}$. This is because using $u$ substitution, we get
    \begin{align*}
        \Gamma\left(\frac{1}{2}\right) &= \int_{0}^{\infty} t^{-\frac{1}{2}}e^{-t} dt \\
                                       &= \int_{0}^{\infty} u^{-1}e^{-u^2}2u du \\
                                       &= 2 \int_{0}^{\infty} e^{-u^2} du \\
                                       &= \int_{-\infty}^{\infty} e^{-u^2} du.
    \end{align*} To evaluate the last integral, we will change to polar coordinates:
    \begin{align*}
        {\left(\int_{-\infty}^{\infty} e^{-u^2} du\right)}^{2} &= \left(\int_{-\infty}^{\infty} e^{-x^2} dx\right)\left(\int_{-\infty}^{\infty} e^{-y^2} dy\right) \\
                                                               &= \int_{-\infty}^{\infty}\int_{-\infty}^{\infty} e^{-(x^2+y^2)} dx dy \\
                                                               &= \int_{0}^{2 \pi}\int_{0}^{\infty} e^{-r^2}r dr d\theta \\
                                                               &= 2 \pi \int_{0}^{\infty} re^{-r^2} dr \\
                                                               &= 2 \pi \int_{-\infty}^{0} \frac{1}{2} e^{s} ds \\
                                                               &= \lim\limits_{x \to -\infty} \pi(e^0 - e^{x}) \\
                                                               &= \pi.
    \end{align*} Therefore, our original integral evaluates to $\sqrt{\pi}$.
}
\newpage
\begin{mybox}
    \theorem{(Cauchy's Residue Theorem) Let $U \subseteq \C$ be a simply connected open set containing a finite number of points $\{a_1, \ldots, a_n\}$. Then, let $U_0$ be $U \setminus \{a_1, \ldots, a_n\}$. Let $f : U_0 \to \C$ be holomorphic on $U_0$. Furthermore, let $\gamma$ be a closed rectifiable curve in $U_0$, $\Res(f, a_i)$ denote the residue of $f$ at each $a_k$, and $I(\gamma, a_k)$ be the winding number of $\gamma$ around $a_k$. Then, \[\oint_{\gamma} f(z) dz = 2 \pi i \sum\limits_{k=1}^{n}I(\gamma, a_k)\Res(f, a_k).\]}
\end{mybox}

\newpage
\end{document}
