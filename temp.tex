%%%%%%%%%%%%%%% Page Setup %%%%%%%%%%%%%%%

\documentclass[a4paper, 11pt]{book}
\usepackage[utf8]{inputenc}
\usepackage[framemethod=tikz]{mdframed}
\usepackage[margin=3cm]{geometry}
\setlength\parindent{0pt}
\setlength{\parskip}{\baselineskip}%
\usepackage{graphicx}
\usepackage{tgadventor}
%\renewcommand*\familydefault{\sfdefault}
\usepackage[T1]{fontenc}

%%%%%%%%%%%%%%% Math Stuff %%%%%%%%%%%%%%%

\usepackage{amsmath}
\usepackage{amssymb}
\usepackage{tikz}
\usepackage{tikz-cd}
\usepackage{tkz-euclide}
\usepackage{pgf-pie}
\usepackage{cancel}

%%%%%%%%%%%%%%% Quotation Stuff %%%%%%%%%%%%%%%

\usepackage[english]{babel}
\usepackage[autostyle]{csquotes}

%%%%%%%%%%%%%%% Color %%%%%%%%%%%%%%%

\definecolor{p}{rgb}{0.9118, 0.4961, 0.6804}
\definecolor{p1}{rgb}{1, 0.92, 0.963}
\definecolor{g}{rgb}{0.023, 0.477, 0.219}
\definecolor{g1}{rgb}{0.03, 0.75, 0.20}
\definecolor{g2}{rgb}{0.95, 1, 0.95}

%%%%%%%%%%%%%%% Link Setup %%%%%%%%%%%%%%%

\usepackage{hyperref}
\hypersetup{colorlinks=true, linkcolor=g, filecolor=g, urlcolor=g,}

%%%%%%%%%%%%%%% Box %%%%%%%%%%%%%%%

\newmdenv[innerlinewidth=2pt, roundcorner=4pt,linecolor=g1,innerleftmargin=20pt,
innerrightmargin=20pt,innertopmargin=20pt,innerbottommargin=20pt,backgroundcolor = g2]{mybox}

%%%%%%%%%%%%%%% Border Stuff %%%%%%%%%%%%%%%

\usepackage{fancyhdr}
\pagestyle{fancy}
\fancyhf{}
\fancyhead[LE,RO]{\leftmark}
\fancyhead[RE,LO]{\thepage}
\fancyfoot[CE,CO]{}
\fancyfoot[LE,RO]{}

%%%%%%%%%%%%%%% Bug Fix Adjustments %%%%%%%%%%%%%%%

\setlength{\headheight}{14pt}
\setlength{\footskip}{55pt}

%%%%%%%%%%%%%%% Index %%%%%%%%%%%%%%%

\usepackage{makeidx}
\makeindex

%%%%%%%%%%%%%%% Custom Commands %%%%%%%%%%%%%%%

\def\greenlozenge{\mathbin{\color{g}\blacklozenge}}

\newcommand{\emphasis}[1]{\underline{\textbf{#1}} }
\newcommand{\vocab}[1]{\underline{\textbf{#1}}\index{#1}}

\newcommand{\claim}{\textbf{\underline{Claim}} }
\newcommand{\corollary}{\underline{\textbf{Corollary}} }
\newcommand{\defn}{\underline{\textbf{Def}} }
\newcommand{\digression}{\underline{\textbf{Digression}} }
\newcommand{\easy}{\underline{\textbf{Easy}} }
\newcommand{\example}{\underline{\textbf{Example}} }
\newcommand{\fact}{\underline{\textbf{Fact}} }
\newcommand{\facts}{\underline{\textbf{Facts}} }
\newcommand{\goal}{\underline{\textbf{Goal}} }
\newcommand{\idea}{\underline{\textbf{Idea}} }
\newcommand{\lemma}{\underline{\textbf{Lemma}} }
\newcommand{\need}{\underline{\textbf{Need}} }
\newcommand{\note}{\underline{\textbf{Note}} }
\newcommand{\proof}{\underline{\textbf{Proof}} }
\newcommand{\recall}{\underline{\textbf{Recall}} }
\newcommand{\remark}{\underline{\textbf{Remark}} }
\newcommand{\theorem}{\underline{\textbf{Theorem}} }
\newcommand{\verify}{\underline{\textbf{Verify}} }

%%%%%%%%%%%%%%% Math Operators %%%%%%%%%%%%%%%
\DeclareMathOperator{\Aut}{Aut}
\DeclareMathOperator{\characteristic}{char}
\DeclareMathOperator{\cod}{cod}
\DeclareMathOperator{\dom}{dom}
\DeclareMathOperator{\Fix}{Fix}
\DeclareMathOperator{\Frac}{Frac}
\DeclareMathOperator{\Free}{Free}
\DeclareMathOperator{\id}{id}
\DeclareMathOperator{\Ima}{Im}
\DeclareMathOperator{\Mor}{Mor}
\DeclareMathOperator{\N}{\mathbb{N}}
\DeclareMathOperator{\Obj}{Obj}
\DeclareMathOperator{\Q}{\mathbb{Q}}
\DeclareMathOperator{\R}{\mathbb{R}}
\DeclareMathOperator{\Stab}{Stab}
\DeclareMathOperator{\Span}{span}
\DeclareMathOperator{\Spec}{Spec}
\DeclareMathOperator{\Z}{\mathbb{Z}}
%%%%%%%%%%%%%%% Title %%%%%%%%%%%%%%%

\title{Transcendental Numbers}
\author{Vincent Lin}
\date{Fall 2023}


\begin{document}
\frontmatter
\maketitle
\tableofcontents
\mainmatter{}
\chapter{Preliminaries}
\section{Transcendental Numbers}
\defn{An \vocab{algebraic number} is a complex number that is the root of a finite nonzero polynomial in one variable with rational coefficients. A \vocab{transcendental number} is a complex number that is not algebraic.}\par

To find some algebraic numbers, we can take a nonzero polynomial with rational coefficents and find its roots. By definition, these roots are algebraic numbers. For example, $\sqrt{2}$ is algebraic because it is a root of $x^2 - 2$. Also, $i$ is algebraic because it is the root of $x^2 + 1$. All rational numbers are algebraic as well. Let $\frac{p}{q} \in \Q$ be rational, where $p, q \in \Z$ and $q$ is nonzero. Then, it is the root of $x - \frac{p}{q}$.\par

What about transcendental numbers? Do they exist?\par

\begin{mybox}
    \theorem{Yes, transcendental numbers exist.}
\end{mybox}

\proof{Consider the set of algebraic numbers. This set is countable. }

\chapter{Lindemann-Weierstrass Theorem}
\begin{mybox}
    \theorem{(Lindemann-Weierstrass Theorem) If $\alpha_1, \ldots, \alpha_n$ algebraic numbers that are linearly independent over $\Q$, then $e^{\alpha_1}, \ldots, e^{\alpha_n}$ are linearly independent over the algebraic numbers. In other words, the extension field $\Q(e^{\alpha_1}, \ldots, e^{\alpha_n})$ has transcendence degree $n$ over $\Q$.}\par
\end{mybox}
\proof{}

\chapter{Gelfond-Schneider Theorem}

\begin{mybox}
    \theorem{(Gelfond-Schneider Theorem) Let $\alpha$ and $\beta$ be algebraic numbers such that $\alpha \notin \{0, 1\}$ and $\beta \in \R \setminus \Q$. Then, ${\alpha}^{\beta}$ is transcendental.}\par
\end{mybox}
\proof{}
\chapter{Weak Form of Baker's Theorem}
\chapter{Schanuel's Conjecture}
\backmatter{}
\printindex
\end{document}
